%!TEX TS-program = xelatex

\documentclass[a4paper,12pt]{article}
%\usepackage[english,russian]{babel} 
\usepackage{fontspec}
\setmainfont[Ligatures=TeX,
						Path=font/,
						BoldFont=brillb,
						ItalicFont=brilli,
						BoldItalicFont=brillbi]{brill}
\setsansfont[Ligatures=TeX,
						Path=font/,
						BoldFont=brillb,
						ItalicFont=brilli,
						BoldItalicFont=brillbi]{brill}

\newfontfamily\tgtermes{TeX Gyre Termes}
\makeatletter
  \begingroup
    \tgtermes
    \DeclareFontShape{\f@encoding}{\rmdefault}{m}{sc}{%
      <-> ssub * \f@family/m/sc}{}
    \DeclareFontShape{\f@encoding}{\rmdefault}{bx}{sc}{%
      <-> ssub * \f@family/bx/sc}{}
  \endgroup
\makeatother

\usepackage{subcaption}
\usepackage{hhline}
\usepackage{enumitem}
\setlist{nolistsep, leftmargin=5mm}
\usepackage{longtable}
\usepackage{multirow} 
\usepackage{multicol} 
\usepackage{setspace} 
%\onehalfspacing % Интерлиньяж 1.5
%\doublespacing % Интерлиньяж 2
\singlespacing % Интерлиньяж 1
\usepackage{geometry} % Простой способ задавать поля
\geometry{top=20mm}
\geometry{bottom=20mm}
\geometry{left=20mm}
\geometry{right=20mm}
\usepackage{hyperref}
\usepackage[usenames,dvipsnames,svgnames,table,rgb]{xcolor}
	\hypersetup{ % Гиперссылки
	colorlinks=true, % false: ссылки в рамках; true: цветные ссылки
	linkcolor=black, % внутренние ссылки
	citecolor=black, % на библиографию
	filecolor=black, % на файлы
	urlcolor=ForestGreen % на URL
	}

\usepackage{forest} % для рисования деревьев
\usepackage{vowel} % для рисования трапеций гласных
\usepackage[toc,page]{appendix}
\usepackage{config/leipzig}
\newleipzig{Abs}{abs}{absolutive case}
\newleipzig{Add}{ad}{Ad location}
\newleipzig{Addi}{add}{additive}
\newleipzig{Ads}{ad2}{Ad2 location}
\newleipzig{Aff}{aff}{affective case}
\newleipzig{An}{an}{animal class}
\newleipzig{Attr}{attr}{attributive particle}
\newleipzig{Aor}{aor}{aorist}
\newleipzig{Caus}{caus}{causative}
\newleipzig{Com}{com}{comitative}
\newleipzig{Dat}{dat}{dative case}
\newleipzig{Erg}{erg}{ergative case}
\newleipzig{Ess}{ess}{essive direction}
\newleipzig{Elat}{elat}{ellative direction}
\newleipzig{F}{f}{female class}
\newleipzig{Gen}{gen}{genitive case}
\newleipzig{H}{h}{human}
\newleipzig{Hab}{hab}{habitual}
\newleipzig{Imp}{imp}{imperative}
\newleipzig{Inst}{inst}{instrumental case}
\newleipzig{Indef}{indef}{indefinite}
\newleipzig{Inter}{inter}{Inter location}
\newleipzig{Jus}{jus}{jussive}
\newleipzig{Lat}{lat}{lative direction}
\newleipzig{M}{m}{masculine class}
\newleipzig{Nanf}{¬an1}{first non-animated class}
\newleipzig{Nans}{¬an2}{second non-animated class}
\newleipzig{Neg}{neg}{negation marker}
\newleipzig{Nh}{¬h}{non-human}
\newleipzig{Npst}{¬pst}{non-past forms}
\newleipzig{Obl}{obl}{oblique stem}
\newleipzig{Pf}{pf}{perfect tense}
\newleipzig{Pl}{pl}{plural}
\newleipzig{Pres}{pres}{present tense}
\newleipzig{Prog}{prog}{progressive}
\newleipzig{Pst}{pst}{past tense}
\newleipzig{Ptcp}{ptcp}{participle}
\newleipzig{Rfl}{refl}{reflexive pronoun}
\newleipzig{Super}{super}{super location}
\newleipzig{Wh}{wh}{special question marker}
\makeglossaries
\usepackage{natbib}
\bibpunct[: ]{[}{]}{;}{a}{}{,}
\usepackage{philex} % пакет для примеров
\renewcommand{\thesection}{\arabic{section}.}
\renewcommand{\thesubsection}{\arabic{section}.\arabic{subsection}}
	\setlength{\columnsep}{1.6cm}
	
\usepackage{footnote}
\makesavenoteenv{tabular}
\title{\Large Relative clause in Zilo}
\author{G. Moroz}
\date{August 2018, \href{https://github.com/agricolamz/2018_Andi_relative_clause/raw/master/main.pdf}{last version}}
\begin{document} 
\maketitle

\section{Introduction}
Zilo speakers uses participles for the formation of relative clauses. Relative clauses can either precede the head or follow the head without any known changes in meaning (see \ref{position-1}--\ref{position-2}), so the topic of the possible word orders in relative clause and whether different orders have different meanings needs further investigation based on corpora data.

\section{Positions that can be relativized}

\ex. Single argument of an intransitive verb
	\ag. iʃːi-χo joʃi j-iʔ-o\\
		 {we-\Add.\Lat} {girl} {\F-come-\Pst(\Aor)}\\
		 \glt    `The? girl came to us'
	\bg. di-<j>o ts'inn-e joʃi [iʃːi-χo j-iʔ-o-b]\\
		 {I-<\F>\Aff} {know-\Hab} {girl} {we-\Add.\Lat} {\F-come-\Pst-\Ptcp:\Pst}\\
		 \glt    `I know the girl who came to us'. \label{position-1}
	\bg. di-<j>o ts'inn-e [iʃːi-χo j-iʔ-o-b] joʃi \\
		 {I-<\F>\Aff} {know-\Hab} {we-\Add.\Lat} {\F-come-\Pst-\Ptcp:\Pst} {girl}\\
		 \glt    `I know the girl who came to us'.
	\bg. joʃi  [iʃːi-χo j-iʔ-o-b] di-<j>o ts'inn-e \\
		  {girl}  {we-\Add.\Lat} {\F-come-\Pst-\Ptcp:\Pst} {I-<\F>\Aff} {know-\Hab}\\
		 \glt    `I know the girl who came to us'.
	\bg. [iʃːi-χo j-iʔ-o-b] joʃi di-<j>o ts'inn-e \\
		  {we-\Add.\Lat} {\F-come-\Pst-\Ptcp:\Pst} {girl} {I-<\F>\Aff} {know-\Hab}\\
		 \glt    `I know the girl who came to us'. \label{position-2}
	\bg. [iʃːi-χo j-iʔ-ija] joʃi di-<j>o ts'inn-e \\
		  {we-\Add.\Lat} {\F-come-\Pst-\Ptcp:\Pres} {girl} {I-<\F>\Aff} {know-\Hab}\\
		 \glt    `I know the girl who will come to us'. 

\ex. Agent of transitive verb
	\ag. maduhalʃ-di χink'il-ol k'amm-i\\
			{neighbour.\Obl-\Erg} {khinkal-\Pl} {eat-\Pst(\Aor)}\\
			\glt `Neighbour ate khinkals'.
	\bg. [χink'il-ol k'amm-i-b] maduhal di-<w>o ts'inn-e \\
			{khinkal-\Pl} {eat-\Pst-\Ptcp:\Pst} neighbour {I-<\M>\Aff} {know-\Hab} \\
		 \glt    `I know the neighbour who ate khinkals'.
	\bg. [χink'il-ol k'amm-ija] maduhal di-<w>o ts'inn-e \\
			{khinkal-\Pl} {eat-\Ptcp:\Pres} neighbour {I-<\M>\Aff} {know-\Hab} \\
		 \glt    `I know the neighbour who will eat khinkals'.

\ex. Patient of transitive verb
	\ag. maduhalʃ-di χink'il-ol k'amm-i\\
			{neighbour.\Obl-\Erg} {khinkal-\Pl} {eat-\Pst(\Aor)}\\
			\glt `Neighbour ate khinkals'.
	\bg. [maduhalʃ-di k'amm-i-b-ol] χink'il-ol den-ni dʒid-i\\
			{neighbour.\Obl-\Erg} {eat-\Pst-\Ptcp:\Pst-\Pl} {khinkal-\Pl}  {I-\Erg} {do-\Pst(\Aor)}\\
		 \glt    `I made khinkals that the neighbour ate'.
	\bg. [maduhalʃ-di k'amm-ij-ol] χink'il-ol den-ni dʒid-i\\
			{neighbour.\Obl-\Erg} {eat-\Ptcp:\Pres-\Pl} {khinkal-\Pl}  {I-\Erg} {do-\Pst(\Aor)}\\
		 \glt    `I made khinkals that the neighbour will eat'.		

\ex. Affective argument
	\ag. dada-<r>o reʃa haʔ-o\\
			{father-<\Nans>\Aff} tree {see-\Pst(\Aor)}\\
			\glt `Father saw the tree'.
	\bg. [reʃa haʔ-o-b] dada di-<w>o ts'inn-e\\
			tree {see-\Pst-\Ptcp:\Pst} father {I-<\M>\Aff} {know-\Hab}\\
			\glt `I know the father who saw the tree'.
	\bg. [reʃa haʔ-ija] dada di-<w>o ts'inn-e\\
			tree {see-\Ptcp:\Pres} father {I-<\M>\Aff} {know-\Hab}\\
			\glt `I know the father who will see the tree'.

\ex. Instrumental argument
	\ag. joʃi haluχ-onn-ij baba-qχi\\
			{girl} {miss-\Pst-\Pf} {mother-\Inst}\\
			\glt `Girl missed her mother'.
	\bg. [joʃi haluχonn-i-b] baba  di-<j>o ts'inn-e\\
			{girl} {miss-\Pst-\Ptcp:\Pst} mother {I-<\M>\Aff} {know-\Hab}\\
			\glt `I know the mother  whom the girl missed'.
	\bg. [joʃi haluχonn-ija] baba  di-<j>o ts'inn-e\\
			{girl} {miss-\Ptcp:\Pres} mother {I-<\M>\Aff} {know-\Hab}\\
			\glt `I know the mother  whom the girl will miss'.

\ex. Ad2 argument
	\ag. woʃo imu-tʃ'u sir-i\\
			{boy} {father.\Obl-\Ads} {afraid-\Pst(\Aor)}\\
			\glt `The boy is afraid of [his] father'.
	\bg. [woʃo sir-i-b] ima di-<w>o ts'inn-e\\
			{boy} {afraid-\Pst-\Ptcp:\Pst} father {I-<\M>\Aff} {know-\Hab} \\
			\glt `I know the father that boy was afraid of'.
	\bg. [woʃo sir-ija] ima di-<w>o ts'inn-e\\
			{boy} {afraid-\Ptcp:\Pres} father {I-<\M>\Aff} {know-\Hab} \\
			\glt `I know the father that boy will afraid of'.

\ex. Dative argument
	\ag. dada-di j-oʃul-ɬu intʃi itʃː-i\\
			{father.\Obl-\Erg} {girl.\Obl-\Dat} apple {give-\Pst(\Aor)}\\
			\glt `Father gave an apple to the daughter'.
	\bg. [dada-di intʃi itʃː-i-b] joʃi di-<j>o ts'inn-e\\
			{father.\Obl-\Erg} apple {give-\Pst-\Ptcp:\Pst} {girl} {I-<\F>\Aff} {know-\Hab} \\
			\glt `I know the girl that father gave an apple to'.
	\bg. [dada-di intʃi itʃː-ija] joʃi di-<j>o ts'inn-e\\
			{father.\Obl-\Erg} apple {give-\Ptcp:\Pres} {girl} {I-<\F>\Aff} {know-\Hab} \\
			\glt `I know the girl that father will give an apple to'.

\ex. Comitative argument
	\ag. joʃi j-ik-on ritɬ'i-loj\\
			girl {\F-live.on-\Pst(\Aor)} {meet-\Com}\\
			\glt `The girl live on meet'.
	\bg. [joʃi j-ik-oni-b] ritɬ'i di-<b>o dʒiʔ-e\\
			girl {\F-live.on-\Pst-\Ptcp:\Pst} meet {I-<\Nanf>\Aff} {love-\Hab}\\
			\glt `I like the meet that girl lived on'.
	\bg. [joʃi j-ik-on-ija] ritɬ'i di-<b>o dʒiʔ-e\\
			girl {\F-live.on-\Npst-\Ptcp:\Pres} meet {I-<\Nanf>\Aff} {love-\Hab}\\
			\glt `I like the meet that girl will live on'.

\ex. Causee argument
	\ag. woʃu-di joʃu-<b>o χink'il-ol k'amm-oɬ-i\\
			{boy.\Obl-\Erg} {girl.\Obl-<\Nanf>\Aff} {khinkal-\Pl} {eat-\Caus-\Pst(\Aor)}\\
			\glt `The girl live on meet'.
	\bg. [woʃu-di χink'il-ol k'amm-oɬ-i-b] joʃi iʃːi-χo j-iʔ-o\\
			{boy.\Obl-\Erg} {khinkal-\Pl} {eat-\Caus-\Pst-\Ptcp:\Pst} girl {we-\Add.\Lat} {\F-come-\Pst(\Aor)}\\
			\glt `The girl who was made to eat khinkals came to us'.
	\bg. [woʃu-di χink'il-ol k'amm-ol-ija] joʃi iʃːi-χo j-iʔ-o\\
			{boy.\Obl-\Erg} {khinkal-\Pl} {eat-\Caus-\Ptcp:\Pres} girl {we-\Add.\Lat} {\F-come-\Pst(\Aor)}\\
			\glt `The girl who will be made to eat khinkals came to us'.
						
\ex. Genitive possessor
	\ag. joʃu-tɬi reʔa ruq'-u-r\\
			{girl.\Obl-\Gen} hand {hurt-\Pst-\Prog}\\
			\glt `Girl's hand hurts'.
	\bg. [reʔa ruq'-u-b] joʃi di-<j>o ts'inn-e\\
			hand {hurt-\Pst-\Ptcp:\Pst} girl {I-<\F>\Aff} {know-\Hab}\\
			\glt `I know the girl whose arm hurt'.
	\bg. [reʔa ruq'-ija] joʃi di-<j>o ts'inn-e\\
			hand {hurt-\Ptcp:\Pres} girl {I-<\F>\Aff} {know-\Hab}\\
			\glt `I know the girl whose arm will hurt'.
			
\ex. Time
	\ag. ʒeɬal woʃo w-uʔ-o\\
			today {boy} {\M-come-\Pst(\Aor)}\\
			\glt `Today boy came'.	
	\bg. [woʃo w-uʔ-o-b] zubu\\
			{boy} {\M-come-\Pst-\Ptcp:\Pst} day\\
			\glt `Day when the boy came'.	
	\bg. [woʃo w-uʔ-ija] zubu\\
			{boy} {\M-come-\Ptcp:\Pres} day\\
			\glt `Day when the boy will come'.	

\ex. Spatial location
	\ag. woʃo hon-tɬi w-uʒ-un\\
			boy {village.\Obl-\Inter.\Ess} {\M-be.born-\Pst(\Aor)}\\
			\glt `The boy was born in a village'.
	\bg. [woʃo w-uʒ-uni-b] hon di-<r>o haʔ-o\\
			boy {\M-be.born-\Pst-\Ptcp:\Pst} village {I-<\Nans>\Aff} {see-\Pst(\Aor)}\\
			\glt `I saw the village, where the boy was born'.
	\bg. [woʃo w-uʒ-un-ija] hon di-<r>o haʔ-o\\
			boy {\M-be.born-\Npst-\Ptcp:\Pres} village {I-<\Nans>\Aff} {see-\Pst(\Aor)}\\
			\glt `I saw the village, where the boy will be born'.
			
Since relativisation remove all relation markers from the head of the relative clause, the ambiguity arise: the relativised spatial goal and spatial source will be the same (but there is a lexical way to distinguish them anyway):

\ex. Spatial goal and spatial source
	\ag. woʃo hon-tɬi w-utʃ'ː-un\\
			boy {village.\Obl-\Inter.\Lat} {\M-run-\Pst(\Aor)}\\
			\glt `Boy ran to the village'.
	\bg. woʃo hon-tɬi-kːu w-utʃ'ː-un\\
			boy {village.\Obl-\Inter-\Elat} {\M-run-\Pst(\Aor)}\\
			\glt `Boy ran from the village'.
	\bg. [woʃo w-utʃ'ː-uni-b] hon di-<r>o haʔ-o\\
			boy {\M-run-\Pst-\Ptcp:\Pst} village {I-<\Nans>\Aff} {see-\Pst(\Aor)}\\
			\glt `I saw the village that boy ran to/from'.
	\bg. [woʃo w-utʃ'ː-un-ija] hon di-<r>o haʔ-o\\
			boy {\M-run-\Npst-\Ptcp:\Pres} village {I-<\Nans>\Aff} {see-\Pst(\Aor)}\\
			\glt `I saw the village that boy will ran to/from'.

In adpositional relative clause the adposition could be placed anywhere in a relative clause but the last position (see \ref{adposition-1}--\ref{adposition-2}). Probably, participle should be placed in last position in a relative clause, but it should be investigated in more accurate way. The first informants' reaction is in example (\ref{adposition-first}).

\ex. Adpositional argument
	\ag. j-oʃu-di k'amm-i intʃi haq'u-tʃ'u tɬeru\\
			{girl.\Obl-\Erg} {eat-\Pst(\Aor)} apple {hous.\Obl-\Ads} near\\
			\glt `The girl ate apple near the hous'.
	\bg. 	[\textbf{tɬeru} j-oʃu-di  intʃi k'amm-i-b] haq'u di-<r>o haʔ-o\\
			near {girl.\Obl-\Erg} apple {eat-\Pst-\Ptcp:\Pst} house  {I-<\Nans>\Aff} {see-\Pst(\Aor)} \label{adposition-1}\\
			\glt `I saw the house near which the girl ate an apple'.
	\bg. 	[j-oʃu-di \textbf{tɬeru} intʃi k'amm-i-b] haq'u di-<r>o haʔ-o\\
			{girl.\Obl-\Erg} near apple {eat-\Pst-\Ptcp:\Pst} house  {I-<\Nans>\Aff} {see-\Pst(\Aor)} \label{adposition-first}\\
			\glt `I saw the house near which the girl ate an apple'.
	\bg. 	[j-oʃu-di intʃi \textbf{tɬeru} k'amm-i-b] haq'u di-<r>o haʔ-o\\
			{girl.\Obl-\Erg} apple near  {eat-\Pst-\Ptcp:\Pst} house  {I-<\Nans>\Aff} {see-\Pst(\Aor)} \\
			\glt `I saw the house near which the girl ate an apple'.
	\bg. 	*[j-oʃu-di intʃi k'amm-i-b \textbf{tɬeru}] haq'u di-<r>o haʔ-o\\
			{girl.\Obl-\Erg} apple {eat-\Pst-\Ptcp:\Pst} near house  {I-<\Nans>\Aff} {see-\Pst(\Aor)} \\
			\glt `I saw the house near which the girl ate an apple'. \label{adposition-2}
	\bg. 	[j-oʃu-di \textbf{tɬeru} intʃi k'amm-ija] haq'u di-<r>o haʔ-o\\
			{girl.\Obl-\Erg} near apple {eat-\Ptcp:\Pres} house  {I-<\Nans>\Aff} {see-\Pst(\Aor)} \\
			\glt `I saw the house near which the girl will eat an apple'.

\ex. Actant from the general question
	\ag. joʃi j-iʔ-o di-bolo dada-di rats'ː-in\\
			{girl} {\F-come-\Pst:(\Aor)} {?-\Indef} {father.\Obl-\Erg} {ask-\Pst(\Aor)}\\
			\glt `Father asked, whether the girl came'.
	\bg. [dada-di  j-iʔ-o di-bolo rats'ː-inni-b] joʃi di-<j>o ts'inn-e-ssu\\
			{father.\Obl-\Erg} {\F-come-\Pst:(\Aor)} {?-\Indef} {ask-\Pst-\Ptcp:\Pst} {girl}  {I-<\F>\Aff} {know-\Hab-\Neg}\\
			\glt `I don't know the girl about whom father asked, whether she came'.
	\bg. [dada-di  j-iʔ-o di-bolo rats'ː-inn-ija] joʃi di-<j>o ts'inn-e-ssu\\
			{father.\Obl-\Erg} {\F-come-\Pst:(\Aor)} {?-\Indef} {ask-\Npst-\Ptcp:\Pres} {girl}  {I-<\F>\Aff} {know-\Hab-\Neg}\\
			\glt `I don't know the girl about whom father will ask, whether she came'.

\ex. Actant from the special question
	\ag. maduhalʃ-χo i-m w-uʔ-o di-bolo dada-di rats'ː-in\\
			{neighbour.obl-\Add.\Lat} {what-\H} {\M-come-\Pst(\Aor)} {?-\Indef} {father.\Obl-\Erg} {ask-\Pst(\Aor)}\\
			\glt `Father asked, who came to the neighbour'.
	\bg. [dada-di i-m w-uʔ-o di-bolo rats'ː-inni-b]  maduhal di-<w>o ts'inn-e-ssu\\
		{father.\Obl-\Erg}  {what-\H} {\M-come-\Pst(\Aor)} {?-\Indef} {ask-\Pst-\Ptcp:\Pst} neighbour {I-<\M>\Aff} {know-\Hab-\Neg}\\
		\glt `I don't know the neighbour to whom father asked who came'.
	\bg. [dada-di i-m w-uʔ-o di-bolo rats'ː-inn-ija]  maduhal di-<w>o ts'inn-e-ssu\\
		{father.\Obl-\Erg}  {what-\H} {\M-come-\Pst(\Aor)} {?-\Indef} {ask-\Npst-\Ptcp:\Pres} neighbour {I-<\M>\Aff} {know-\Hab-\Neg}\\
		\glt `I don't know the neighbour to whom father asked who came'.

\ex. So called relative clause without coreferential noun phrase in the modifying clause
	\ag. [den-ni zĩw b-ertʃ'ː-i-b] ʃːiw woʃu-di ts'ad-i\\
			{I-\Erg} cow {\An-milk-\Pst-\Ptcp:\Pst} milk {boy.\Obl-\Erg} {drink-\Pst(\Aor)}\\
			\glt `Boy drank the milk from my milking the cow'.
	\bg. [den-ni maʃina b-aχo-ɬ-i-b] orsi di-<b>o haʔ-o-ssu\\
			 {I-\Erg}  car {\Nanf-sell-\Caus-\Pst-\Ptcp:\Pst} money {I-<\Nanf>\Aff} {see-\Pst(\Aor)-\Neg}\\
			 \glt `I didn't see the money from my selling the car'.
	\bg. [woʃul-di orsi b-eq'aʃː-i-b] χabar di-<b>o tsinn-e\\
			{boy.\Obl-\Erg} money {\Nanf-steal-\Pst-\Ptcp:\Pst} story {I-<\Nanf>\Aff} {know-\Hab}\\
			\glt `I know the story about the boy stealing the money'
	\bg. [joʃu-di χʷami-l j-eʒ-a-b] ʃam di-<b>o b-aχo-r\\
			{girl.\Obl-\Erg} {fish-\Pl} {\F-cook-\Pst-\Ptcp:\Pst} smell {I-<\Nanf>\Aff} {\Nanf-smell-\Prog}\\
			\glt `I pick up the smell of fishes that girl cooked'.
			 
It is not possible to relativise some non-obligatory adjuncts:

\ex. Commitative adjunct
	\ag. dada joʃi-loj w-uʔ-o\\
			father {girl-\Com} {\M-come-\Pst(\Aor)}\\
			\glt `Father came with the daughter'.	
	\bg. *[dada w-uʔ-o-b] joʃi di-<j>o ts'inn-e\\
			father {\M-come-\Pst-\Ptcp:\Pst} girl {I-<\F>\Aff} {know-\Hab}\\
			\glt `I know the daughter that father came with'.

It is not possible to relativise from coordination island:

\ex. Coordination Island
	\ag. woʃo-lo joʃi-lo w-uʔ-o\\
			{boy-\Addi} {girl-\Addi} {\M-come-\Pst(\Aor)}\\
			\glt `A boy and a girl came'.	
	\bg. *[woʃo-lo w-uʔ-o-b] joʃi di-<j>o ts'inn-e\\
			{boy-\Addi} {\M-come-\Pst-\Ptcp:\Pst} girl {I-<\F>\Aff} {know-\Hab}\\
			\glt `I know the girl that boy and she came'.

\section{Head properties}
The head of the relative clause could be:

\ex. common noun
	\ag. [iʃːi-χo j-iʔ-o-b] \textbf{joʃi} di-<j>o ts'inn-e \\
		  {we-\Add.\Lat} {\F-come-\Pst-\Ptcp:\Pst} {girl} {I-<\F>\Aff} {know-\Hab}\\
		 \glt    `I know the girl who came to us'.

\ex. demonstrative pronoun
	\ag. [hon-tɬi w-uʒ-uni-b] \textbf{heɡe-w} hoɬu w-etʃ'uχa w-uʔ-un\\
			{village.\Obl-\Inter.\Ess} {\M-be.born-\Pst-\Ptcp:\Pst} {dem-\M} here {\M-big} {\M-become-\Pst(\Aor)}\\
			\glt `He, that was born in village, grew up here'.

\ex. personal pronoun
	\ag. [hon-tɬi w-uʒ-uni-b] \textbf{men} hoɬu w-etʃ'uχa w-uʔ-un\\
			{village.\Obl-\Inter.\Ess} {\M-be.born-\Pst-\Ptcp:\Pst} you here {\M-big} {\M-become-\Pst(\Aor)}\\
			\glt `You, that was born in village, grew up here'.
			
\ex. interrogative pronoun
	\ag. [iʃːi-χo w-uʔ-o-b] i-m-ʁi hede-w?\\
			{we-\Add.\Lat} {\M-come-\Pst-\Ptcp.\Pst} {what-\H-\Wh} {??-\M}\\
			\glt `Who was that who came to us?'
	\bg. [maduhalʃ-di k'amm-i-b] i-b-ʁi hede-b?\\
			{neighbour.\Obl-\Erg} {eat-\Pst-\Ptcp:\Pst} {what-\Nh-\Wh} {??-\Nanf}\\
			\glt `What was that that the neighbour ate?'
		
\ex. indefinite pronoun
	\ag.  [iʃːi-χo w-uʔ-o-b] i-m-bolo di-<w>o  ts'inn-e-ssu\\
			{we-\Add.\Lat} {\M-come-\Pst-\Ptcp.\Pst} {what-\H-\Indef} {I-<\M>\Aff} {know-\Hab-\Neg}\\
			\glt `I don't know somebody who came to us'.
	\bg.  [maduhalʃ-di k'amm-i-b] i-bolo di-<b>o  ts'inn-e-ssu\\
			{neighbour.\Obl-\Erg} {eat-\Pst-\Ptcp:\Pst} {what-\Indef} {I-<\Nanf>\Aff} {know-\Hab-\Neg}\\
			\glt `I don't know something that neighbour ate'.

\ex. heavy head
	\ag.  [iʃːi-χo j-iʔ-o-b] j-etʃ'uχa j-otsːi aminat di-<j>o dʒiʔ-e\\
			{we-\Add.\Lat} {\F-come-\Pst-\Ptcp.\Pst} {\F-big} {\F-sibling} Aminat {I-<\F>\Aff} {love-\Hab}\\
			\glt `I love older sister Aminat, that came to us'.

\ex. headless
	\ag.  [iʃːi-χo j-iʔ-o-b] di-<j>o ts'inn-e-ssu\\
			{we-\Add.\Lat} {\F-come-\Pst-\Ptcp.\Pst} {I-<\F>\Aff} {know-\Hab-\Neg}\\
			\glt `I don't know (female) that came to us'.
	\bg.  [χink'il-ol k'amm-i-b] di-<w>o ts'inn-e-ssu\\
			{khinkal-\Pl} {eat-\Pst-\Ptcp:\Pst} {I-<\M>\Aff} {know-\Hab-\Neg}\\
			\glt `I don't know (male) who ate khinkals'.

\section{Properties of the participle}
There are several categories that could be expressed on participle:

\ex. number
	\ag. [maduhalʃ-di k'amm-i-b] χink'il den-ni dʒid-i\\
			{neighbour.\Obl-\Erg} {eat-\Pst-\Ptcp:\Pst} {khinkal}  {I-\Erg} {do-\Pst(\Aor)}\\
		 \glt    `I made khinkal that the neighbour ate'.
	\bg. [maduhalʃ-di k'amm-ija] χink'il den-ni dʒid-i\\
			{neighbour.\Obl-\Erg} {eat-\Ptcp:\Pres} {khinkal}  {I-\Erg} {do-\Pst(\Aor)}\\
		 \glt    `I made khinkal that the neighbour will eat'.	
	\bg. [maduhalʃ-di k'amm-i-b-ol] χink'il-ol den-ni dʒid-i\\
			{neighbour.\Obl-\Erg} {eat-\Pst-\Ptcp:\Pst-\Pl} {khinkal-\Pl}  {I-\Erg} {do-\Pst(\Aor)}\\
		 \glt    `I made khinkals that the neighbour ate'.
	\bg. [maduhalʃ-di k'amm-ij-ol] χink'il-ol den-ni dʒid-i\\
			{neighbour.\Obl-\Erg} {eat-\Ptcp:\Pres-\Pl} {khinkal-\Pl}  {I-\Erg} {do-\Pst(\Aor)}\\
		 \glt    `I made khinkals that the neighbour will eat'.	

\ex. negation
	\ag. [iʃːi-χo j-iʔ-o-ssu-b] joʃi di-<j>o ts'inn-e \\
		  {we-\Add.\Lat} {\F-come-\Pst-\Neg-\Ptcp:\Pst} {girl} {I-<\F>\Aff} {know-\Hab}\\
		 \glt    `I know the girl who didn't come to us'.
	\bg. [iʃːi-χo j-iʔ-essa] joʃi di-<j>o ts'inn-e \\
		  {we-\Add.\Lat} {\F-come-\Neg.\Ptcp:\Pres} {girl} {I-<\F>\Aff} {know-\Hab}\\
		 \glt    `I know the girl who won't come to us'.
		 
\ex. class agreement (only with absolutive from relative clause)
	\ag. [ħajman \textbf{b}-iqχ-u-b] woʃo di-<w>o ts'inn-e \\
			ram {\Nanf-slaughter-\Pst-\Ptcp:\Pst} boy {I-<\M>\Aff} {know-\Hab}\\
			\glt `I know the boy who slaughter the ram'.
	\bg. [woʃul-di \textbf{b}-iqχ-u-b] ħajman di-<w>o ts'inn-e \\
			{boy.\Obl-\Erg} {\Nanf-slaughter-\Pst-\Ptcp:\Pst} ram {I-<\M>\Aff} {know-\Hab}\\
			\glt `I know the ram that boy slaughtered'.	
			
\ex. case (only on headless relative clauses)
	\ag. [iʃːi-χo j-iʔ-o-b]-di χink'il-ol k'amm-i\\
			{we-\Add.\Lat} {\M-come-\Pst-\Ptcp.\Pst-\Erg} {khinkal-\Pl} {eat-\Pst(\Aor)}\\
	\bg. [iʃːi-χo j-iʔ-o-b] joʃi-di χink'il-ol k'amm-i\\
			{we-\Add.\Lat} {\M-come-\Pst-\Ptcp.\Pst-\Erg} {girl.\Obl-\Erg} {khinkal-\Pl} {eat-\Pst(\Aor)}\\
	\bg. *[iʃːi-χo j-iʔ-o-b]-di joʃi-di χink'il-ol k'amm-i\\
			{we-\Add.\Lat} {\M-come-\Pst-\Ptcp.\Pst-\Erg} {girl.\Obl-\Erg} {khinkal-\Pl} {eat-\Pst(\Aor)}\\
	\bg. *[iʃːi-χo j-iʔ-o-b]-di joʃi χink'il-ol k'amm-i\\
			{we-\Add.\Lat} {\M-come-\Pst-\Ptcp.\Pst-\Erg} {girl.\Obl-\Erg} {khinkal-\Pl} {eat-\Pst(\Aor)}\\
			\glt `(The girl), who came to us, ate khinkals'.

\section{Resumptives}
During the elicitation tasks native speakers didn't produce any resumptives by themselves. Even if informants allow to use reflexive pronouns as a resumptives, they still find this sentences little bit strange (\ref{rfl-resumptive}). No attempts to use demonstratives as a resumptive succeeded. In example (\ref{dem-resumptive}) demonstrative tend to be a modifier of relative clause head \textit{joʃi}, but more work should be done on this topic. In the most natural example (\ref{resumptive?}) unfortunately it is not clear, whether \textit{enʃuro} is in relative clause or it is a head of the NP with a relative clause modifier. This should be investigated more precisely, when the Zilo corpora will be available.

\ex.  
	\ag. $^?$[\textbf{enʃu-<r>o} reʃa haʔ-o-b] dada iʃːi-χo w-uʔ-o\\
		{\Rfl.\Obl-<\Nans>\Aff} tree {see-\Pst-\Ptcp:\Pst} father {we-\Add.\Lat} {\M-come-\Pst(\Aor)}\\
		\glt `Father that saw a tree came to us' \label{rfl-resumptive}
	\bg. \textbf{heɡe-j} [iʃːi-χo j-iʔ-o-b] joʃi di-<j>o ts'inn-e\\
		{\Dem-\F} {we-\Add.\Lat} {\M-come-\Pst-\Ptcp:\Pst} girl {I-<\F>\Aff} {know-\Hab}\\
		\glt `I know this girl who came to us'. \label{dem-resumptive}
	\bg. \textbf{enʃu-<r>o} reʃa haʔ-o-b iʃːi-χo w-uʔ-o-do\\
		{\Rfl.\Obl-<\Nans>\Aff} tree {see-\Pst-\Ptcp:\Pst} {we-\Add.\Lat} {\M-come-\Imp-\Jus}\\
		\glt `That who saw a tree let's come to us!' \label{resumptive?}
 
\section*{Glosses}
\small
\printglosses
\bibliographystyle{config/chicago}
\bibliography{config/bibliography}
\normalsize
\end{document}