\input{config/pre}
\title{\Large Relative clause in Zilo}
\author{G. Moroz}
\date{\small 08.2018}
\begin{document} 
\maketitle

\section{Introduction}
Zilo speakers uses participles for the formation of relative clauses. Relative clauses either precede the head or follow the head without any changes in meaning (see \ref{position-1}--\ref{position-2}), so the topic of the word order in relative clause needs further investigation based on corpora data.

\section{Positions that can be relativized}

\ex. Single argument of an intransitive verb
	\ag. iʃːi-χo joʃi j-iʔ-o\\
		 {we-\Add.\Lat} {girl} {\F-come-\Pst(\Aor)}\\
		 \glt    `The? girl came to us'
	\bg. di-<j>o ts'inn-e joʃi [iʃːi-χo j-iʔ-o-b]\\
		 {I-<\F>\Aff} {know-\Hab} {girl} {we-\Add.\Lat} {\F-come-\Pst-\Ptcp:\Pst}\\
		 \glt    `I know the girl who came to us'. \label{position-1}
	\bg. di-<j>o ts'inn-e [iʃːi-χo j-iʔ-o-b] joʃi \\
		 {I-<\F>\Aff} {know-\Hab} {we-\Add.\Lat} {\F-come-\Pst-\Ptcp:\Pst} {girl}\\
		 \glt    `I know the girl who came to us'.
	\bg. joʃi  [iʃːi-χo j-iʔ-o-b] di-<j>o ts'inn-e \\
		  {girl}  {we-\Add.\Lat} {\F-come-\Pst-\Ptcp:\Pst} {I-<\F>\Aff} {know-\Hab}\\
		 \glt    `I know the girl who came to us'.
	\bg. [iʃːi-χo j-iʔ-o-b] joʃi di-<j>o ts'inn-e \\
		  {we-\Add.\Lat} {\F-come-\Pst-\Ptcp:\Pst} {girl} {I-<\F>\Aff} {know-\Hab}\\
		 \glt    `I know the girl who came to us'. \label{position-2}
	\bg. [iʃːi-χo j-iʔ-ija] joʃi di-<j>o ts'inn-e \\
		  {we-\Add.\Lat} {\F-come-\Pst-\Ptcp:\Prae} {girl} {I-<\F>\Aff} {know-\Hab}\\
		 \glt    `I know the girl who will come to us'. 

\ex. Agent of transitive verb
	\ag. maduhalʃ-di χink'il-ol k'amm-i\\
			{neighbour.\Obl-\Erg} {khinkal-\Pl} {eat-\Pst(\Aor)}\\
			\glt `Neighbour ate khinkals'.
	\bg. [χink'il-ol k'amm-i-b] maduhal di-<w>o ts'inn-e \\
			{khinkal-\Pl} {eat-\Pst-\Ptcp:\Pst} neighbour {I-<\M>\Aff} {know-\Hab} \\
		 \glt    `I know the neighbour who ate khinkals'.
	\bg. [χink'il-ol k'amm-ija] maduhal di-<w>o ts'inn-e \\
			{khinkal-\Pl} {eat-\Ptcp:\Prae} neighbour {I-<\M>\Aff} {know-\Hab} \\
		 \glt    `I know the neighbour who will eat khinkals'.

\ex. Patient of transitive verb
	\ag. maduhalʃ-di χink'il-ol k'amm-i\\
			{neighbour.\Obl-\Erg} {khinkal-\Pl} {eat-\Pst(\Aor)}\\
			\glt `Neighbour ate khinkals'.
	\bg. [maduhalʃ-di k'amm-i-b-ol] χink'il-ol den-ni dʒid-i\\
			{neighbour.\Obl-\Erg} {eat-\Pst-\Ptcp:\Pst-\Pl} {khinkal-\Pl}  {I-\Erg} {do-\Pst(\Aor)}\\
		 \glt    `I made khinkals that the neighbour ate'.
	\bg. [maduhalʃ-di k'amm-ij-ol] χink'il-ol den-ni dʒid-i\\
			{neighbour.\Obl-\Erg} {eat-\Ptcp:\Prae-\Pl} {khinkal-\Pl}  {I-\Erg} {do-\Pst(\Aor)}\\
		 \glt    `I made khinkals that the neighbour will eat'.		

\ex. Affective argument
	\ag. dada-<r>o reʃa haʔ-o\\
			{father-<\Nans>\Aff} tree {see-\Pst(\Aor)}\\
			\glt `Father saw the tree'.
	\bg. [reʃa haʔ-o-b] dada di-<w>o ts'inn-e\\
			tree {see-\Pst-\Ptcp:\Pst} father {I-<\M>\Aff} {know-\Hab}\\
			\glt `I know the father who saw the tree'.
	\bg. [reʃa haʔ-ija] dada di-<w>o ts'inn-e\\
			tree {see-\Ptcp:\Prae} father {I-<\M>\Aff} {know-\Hab}\\
			\glt `I know the father who will see the tree'.

\ex. Instrumental argument
	\ag. joʃi haluχ-onn-ij baba-qχi\\
			{girl} {miss-\Pst-\Pf} {mother-\Inst}\\
			\glt `Girl missed her mother'.
	\bg. [joʃi haluχonn-i-b] baba  di-<j>o ts'inn-e\\
			{girl} {miss-\Pst-\Ptcp:\Pst} mother {I-<\M>\Aff} {know-\Hab}\\
			\glt `I know the mother  whom the girl missed'.
	\bg. [joʃi haluχonn-ija] baba  di-<j>o ts'inn-e\\
			{girl} {miss-\Ptcp:\Prae} mother {I-<\M>\Aff} {know-\Hab}\\
			\glt `I know the mother  whom the girl missed'.

\ex. Ad2 argument
	\ag. woʃo imu-tʃ'u sir-i\\
			{boy} {father.\Obl-\Ads} {afraid-\Pst(\Aor)}\\
			\glt `The boy is afraid of [his] father'.
	\bg. [woʃo sir-i-b] ima di-<w>o ts'inn-e\\
			{boy} {afraid-\Pst-\Ptcp:\Pst} father {I-<\M>\Aff} {know-\Hab} \\
			\glt `I know the father that boy was afraid of'.
	\bg. [woʃo sir-ija] ima di-<w>o ts'inn-e\\
			{boy} {afraid-\Ptcp:\Prae} father {I-<\M>\Aff} {know-\Hab} \\
			\glt `I know the father that boy will afraid of'.

\ex. Dative argument
	\ag. dada-di j-oʃul-ɬu intʃi itʃː-i\\
			{father.\Obl-\Erg} {girl.\Obl-\Dat} apple {give-\Pst(\Aor)}\\
			\glt `Father gave an apple to the daughter'.
	\bg. [dada-di intʃi itʃː-i-b] joʃi di-<j>o ts'inn-e\\
			{father.\Obl-\Erg} apple {give-\Pst-\Ptcp:\Pst} {girl} {I-<\F>\Aff} {know-\Hab} \\
			\glt `I know the girl that father gave an apple to'.
	\bg. [dada-di intʃi itʃː-ija] joʃi di-<j>o ts'inn-e\\
			{father.\Obl-\Erg} apple {give-\Ptcp:\Prae} {girl} {I-<\F>\Aff} {know-\Hab} \\
			\glt `I know the girl that father will give an apple to'.
			
\ex. Time
	\ag. ʒeɬal woʃo w-uʔ-o\\
			today {boy} {\M-come-\Pst(\Aor)}\\
			\glt `Today boy came'.	
	\bg. woʃo w-uʔ-o-b zubu\\
			{boy} {\M-come-\Pst-\Ptcp:\Pst} day\\
			\glt `Day when the boy came'.	
	\bg. woʃo w-uʔ-ija zubu\\
			{boy} {\M-come-\Ptcp:\Prae} day\\
			\glt `Day when the boy will come'.	

\ex. Spatial location

\ex. Spatial goal

\ex. Spatial source

In adpositional relative clause the adposition could be placed anywhere in a relative clause but the last position (see \ref{adposition-1}--\ref{adposition-2}). Probably, participle should be placed in last position in a relative clause, but it should be investigated in more accurate way. The first informants' reaction is in example (\ref{adposition-first}).

\ex. Adpositional argument
	\ag. j-oʃu-di k'amm-i intʃi haq'u-tʃ'u tɬeru\\
			{girl.\Obl-\Erg} {eat-\Pst(\Aor)} apple {hous.\Obl-\Ads} near\\
			\glt `The girl ate apple near the hous'.
	\bg. 	[\textbf{tɬeru} j-oʃu-di  intʃi k'amm-i-b] haq'u di-<r>o haʔ-o\\
			near {girl.\Obl-\Erg} apple {eat-\Pst-\Ptcp:\Pst} house  {I-<\Nans>\Aff} {see-\Pst(\Aor)} \label{adposition-1}\\
			\glt `I saw the house near which the girl ate an apple'.
	\bg. 	[j-oʃu-di \textbf{tɬeru} intʃi k'amm-i-b] haq'u di-<r>o haʔ-o\\
			{girl.\Obl-\Erg} near apple {eat-\Pst-\Ptcp:\Pst} house  {I-<\Nans>\Aff} {see-\Pst(\Aor)} \label{adposition-first}\\
			\glt `I saw the house near which the girl ate an apple'.
	\bg. 	[j-oʃu-di intʃi \textbf{tɬeru} k'amm-i-b] haq'u di-<r>o haʔ-o\\
			{girl.\Obl-\Erg} apple near  {eat-\Pst-\Ptcp:\Pst} house  {I-<\Nans>\Aff} {see-\Pst(\Aor)} \\
			\glt `I saw the house near which the girl ate an apple'.
	\bg. 	*[j-oʃu-di intʃi k'amm-i-b \textbf{tɬeru}] haq'u di-<r>o haʔ-o\\
			{girl.\Obl-\Erg} apple {eat-\Pst-\Ptcp:\Pst} near house  {I-<\Nans>\Aff} {see-\Pst(\Aor)} \\
			\glt `I saw the house near which the girl ate an apple'. \label{adposition-2}
	\bg. 	[j-oʃu-di \textbf{tɬeru} intʃi k'amm-ija] haq'u di-<r>o haʔ-o\\
			{girl.\Obl-\Erg} near apple {eat-\Ptcp:\Prae} house  {I-<\Nans>\Aff} {see-\Pst(\Aor)} \\
			\glt `I saw the house near which the girl will eat an apple'.

\ex. Genitive

\ex. Actant from subordinate clause

\ex. Actant from the general question
	\ag. joʃi j-iʔ-o dibolo dada-di rats'ː-in\\
			{girl} {\F-come-\Pst:(\Aor)} ? {father.\Obl-\Erg} {ask-\Pst(\Aor)}\\
			\glt `Father asked, whether the girl came'.
	\bg. [dada-di  j-iʔ-o dibolo rats'ː-ini-b] joʃi di-<j>o ts'inn-e-ssu\\
			{father.\Obl-\Erg} {\F-come-\Pst:(\Aor)} ? {ask-\Pst-\Ptcp:\Pst} {girl}  {I-<\F>\Aff} {know-\Hab-\Neg}\\
			\glt `I don't know the girl about whom father asked, whether she came'.
	\bg. [dada-di  j-iʔ-o dibolo rats'ː-ini-ja] joʃi di-<j>o ts'inn-e-ssu\\
			{father.\Obl-\Erg} {\F-come-\Pst:(\Aor)} ? {ask-\Npst-\Ptcp:\Prae} {girl}  {I-<\F>\Aff} {know-\Hab-\Neg}\\
			\glt `I don't know the girl about whom father will ask, whether she came'.

\ex. Actant from the special question
	\ag. maduhalʃ-χo im w-uʔ-o dibolo dada-di rats'ː-in\\
			{neighbour.obl-\Add.\Lat} who {\M-come-\Pst(\Aor)} ? {father.\Obl-\Erg} {ask-\Pst(\Aor)}\\
			\glt `Father asked, who came to the neighbour'.
	\bg. [dada-di im w-uʔ-o dibolo rats'ː-ini-b]  maduhal di-<w>o ts'inn-e-ssu\\
		{father.\Obl-\Erg}  who {\M-come-\Pst(\Aor)} ?  {ask-\Pst-\Ptcp:\Pst} neighbour {I-<\F>\Aff} {know-\Hab-\Neg}\\
		\glt `I don't know the neighbour to whom father asked who came'.
	\bg. [dada-di im w-uʔ-o dibolo rats'ː-ini-ja]  maduhal di-<w>o ts'inn-e-ssu\\
		{father.\Obl-\Erg}  who {\M-come-\Pst(\Aor)} ?  {ask-\Npst-\Ptcp:\Prae} neighbour {I-<\F>\Aff} {know-\Hab-\Neg}\\
		\glt `I don't know the neighbour to whom father asked who came'.

The are several cases, when relativisation is not possible:

\ex. Commitative adjunct
	\ag. dada joʃi-loj w-uʔ-o\\
			father {girl-\Com} {\M-come-\Pst(\Aor)}\\
			\glt `Father came with the daughter'.	
	\bg. *[dada w-uʔ-o] joʃi di-<j>o ts'inn-e\\
			father {girl-\Com} {\M-come-\Pst(\Aor)} {I-<\F>\Aff} {know-\Hab}\\
			\glt `I know the daughter that father came with'.

\ex. Coordination Island
	\ag. woʃo-lo joʃi-lo w-uʔ-o\\
			{boy-\Addi} {girl-\Addi} {\M-come-\Pst(\Aor)}\\
			\glt `A boy and a girl came'.	
	\bg. *[woʃo-lo w-uʔ-o] joʃi di-<j>o ts'inn-e\\
			{boy-\Addi} {\M-come-\Pst(\Aor)} girl {I-<\F>\Aff} {know-\Hab}\\
			\glt `I know the girl that boy and she came'.

\section{Head properties}
The head of the relative clause could be:

\ex. Common noun
	\ag. [iʃːi-χo j-iʔ-o-b] joʃi di-<j>o ts'inn-e \\
		  {we-\Add.\Lat} {\F-come-\Pst-\Ptcp:\Pst} {girl} {I-<\F>\Aff} {know-\Hab}\\
		 \glt    `I know the girl who came to us'.

\ex. demonstrative pronoun

\ex. personal pronoun

\ex. interrogative pronoun

\ex. indefinite pronoun

\ex. personal name

\ex. headless

\ex. heavy head

\section{Properties of the participle}
There are several categories that could be expressed on participle:

\ex. number
	\ag. [maduhalʃ-di k'amm-i-b] χink'il den-ni dʒid-i\\
			{neighbour.\Obl-\Erg} {eat-\Pst-\Ptcp:\Pst} {khinkal}  {I-\Erg} {do-\Pst(\Aor)}\\
		 \glt    `I made khinkal that the neighbour ate'.
	\bg. [maduhalʃ-di k'amm-ija] χink'il den-ni dʒid-i\\
			{neighbour.\Obl-\Erg} {eat-\Ptcp:\Prae} {khinkal}  {I-\Erg} {do-\Pst(\Aor)}\\
		 \glt    `I made khinkal that the neighbour will eat'.	
	\bg. [maduhalʃ-di k'amm-i-b-ol] χink'il-ol den-ni dʒid-i\\
			{neighbour.\Obl-\Erg} {eat-\Pst-\Ptcp:\Pst-\Pl} {khinkal-\Pl}  {I-\Erg} {do-\Pst(\Aor)}\\
		 \glt    `I made khinkals that the neighbour ate'.
	\bg. [maduhalʃ-di k'amm-ij-ol] χink'il-ol den-ni dʒid-i\\
			{neighbour.\Obl-\Erg} {eat-\Ptcp:\Prae-\Pl} {khinkal-\Pl}  {I-\Erg} {do-\Pst(\Aor)}\\
		 \glt    `I made khinkals that the neighbour will eat'.	

\ex. negation

\ex. class agreement

\ex. case


\section*{Glosses}
\small
\printglosses
\bibliographystyle{config/chicago}
\bibliography{config/bibliography}
\normalsize
\end{document}